% Evaluation

\chapter{Evaluation (1-2)} % Main chapter title

\label{Chapter4} % For referencing the chapter elsewhere, use \ref{Chapter4} 

\lhead{Chapter 4. \emph{Evaluation}} % This is for the header on each page - perhaps a shortened title

%----------------------------------------------------------------------------------------

Game developers use personal preferences and creative programming techniques and tools to develop games with the hopes of successful market penetration. Often, in the course of development, the needs of the end user are lost. 

Evaluation can occur during various times during the design and development life cycle of a game – early, in the middle, late, and at the end. However, not all types of evaluation methods can be applied during all phases of design and development.

\section{Formative}
\subsection{Single-Condition Study}
 - students \\
 - measure learning, affect, liking \\
 - good for evaluation while implementing, not really for the end product \\
 \\
Were the mechanics easy to understand? \\
How many time did you need to learn how to play the game? \\
Did you find the game challenging? \\
Did you finish the game? \\
How long did it take before you got bored of playing the game? \\
Did you find the user interface intuitive and easy to use? \\
Did you empathise with the main character? \\

 
 
\subsection{Comparison to Original Songs}
\section{Summative}
\subsection{Comparison to Original Songs}
\subsection{time spent tracking}
\subsection{Evaluation Framework}
\subsection{Release}
4.2.2. interviews, questionnaires, surveys
