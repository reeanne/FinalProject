% Evaluation

\chapter{Evaluation} % Main chapter title

\label{Chapter4} % For referencing the chapter elsewhere, use \ref{Chapter4} 

\lhead{Chapter 4. \emph{Evaluation}} % This is for the header on each page - perhaps a shortened title

%----------------------------------------------------------------------------------------

Game developers use personal preferences and creative programming techniques and tools to develop games with the hopes of successful market penetration. Often, in the course of development, the needs of the end user are lost. 

Evaluation can occur during various times during the design and development life cycle of a game – early, in the middle, late, and at the end. However, not all types of evaluation methods can be applied during all phases of design and development.

\section{Formative}
\subsection{Single-Condition Study}
Throughout the course of the design and development of the game we will be conducting studies by asking small groups of people to play our game. The main aims of this type of study is to learn about the opinion the game causes and to observe reactions of the players while they are testing it. This helps avoiding people being biased. We would like to observe the pace at which they learn the rules without being instructed in person, telling us whether the user interface is intuitive, if they find the game challenging, which would be visible in their scores and their engagement (do they try different songs over and over again or do they get bored after 10 minutes of playing?). 

\section{Summative}
\subsection{Comparison to Original Songs}
In order to evaluate the quality of the gameplay generated by our program, we will test our game with songs already existing in the original Guitar Hero game and compare the output we get with its already defined levels. However, to make that possible, the music track we feed to our program must be an instrumental version of the same song as Guitar Hero's songs are mapped onto the key presses by looking at the guitar line of the song, not the main melody. We can then create statistics of correctly identified, false alarm and missed buttons.

\subsection{Melody Extraction Testing}
Another way of evaluating the game is creating a set of songs and generating levels for them. After that a trained Guitar Hero player can play those levels. If the buttons were consistently on time with the notes then the melody extraction and game synchronisation techniques are considered to work.


\subsection{Evaluation Framework}
\subsection{Release}
4.2.2. interviews, questionnaires, surveys
