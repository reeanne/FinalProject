% Introduction

\chapter{Introduction} % Main chapter title

\label{Chapter1} % For referencing the chapter elsewhere, use \ref{Chapter1} 

\lhead{Chapter 1. \emph{Introduction}} % This is for the header on each page - perhaps a shortened title

%----------------------------------------------------------------------------------------

\begin{quotation}
Music and games share a fundamental property: both are playable, offering their listeners and operators an expressive experience with the framework of melody and rhythm \cite{introquote}.
\end{quotation} 

As the quote suggests, both games and music have one thing in common — the act of playing. Just as player’s character might die in an attempt to complete a level, causing him to lose the game, the pianist can fail at the attempt of performing a musical piece. 

Perhaps this analogy inspired programmers to develop a new genre of games - music games. Music games are games in which players interact with music. Possibly the most commonly known franchises in this genre are Guitar Hero, Rock Band and Dance Dance Revolution. In this type of games user has to follow the indicators on the screen telling him which buttons to hit. 

The concept of a music game stormed the industry in 2005, after Guitar Hero was released. The project soon turned into the fastest new video game franchise to reach \$1 billion in retail sales in the history of the business, with Guitar Hero III being the first game to reach \$1 billion \cite{GHSales}.

However, a limited amount of songs transcribed and adjusted to the game play soon caused the popularity of such music video games to decline. Some brave fans of the franchises took it upon themselves to transcribe songs to create new levels. The producers, seeing the tendency, started releasing the in-app purchases to enable the players to extend their library and thus, keep the users. 

Due to the time consuming and difficult nature of the process of manually adding new songs, most players usually limit themselves to pre-processed songs provided by the game producers, not really taking advantage of the full capabilities of the games. 

This project aims to change the way users look at the music rhythm games. We are creating a game which will allow them to upload any song they would like and automatically generate a Guitar Hero-like level corresponding to it. 

This will be achieved by implementation of a melody extraction from polyphonic music signals algorithm using pitch contour characterisation. The algorithm consists of four parts - sinusoid extraction, salience function, pitch contour creation and melody selection. In this approach, pitch contours - time continuous sequences of pitch candidates, are grouped using auditory streaming cues. To filter them, we define a set of contour characteristics, which help distinguish between melodic and non-melodic contours. This leads to the development of new voicing detection, octave error minimisation and melody selection techniques \cite{salamon}.

We will then design and develop an algorithm for mapping the extracted to a series of buttons on the screen to create an interesting and challenging game for a user, as no literature describing such problem was found so far.

In addition to this, we will attempt to develop a mood extraction algorithm to dynamically generate surroundings in the game. Specifically, we treat music emotion recognition as a regression problem to predict the arousal and valence values (AV values) of each music sample directly, which then can be used to generate unique surroundings for every level generated. This continuous view of music emotion makes the proposed music emotion recognition system free of the inherent ambiguity issue. In addition to this, because there is more freedom in describing a song compared to defining and assigning mood classes, the subjectivity issue is alleviated to some extent. \cite{mood}.

With this project we would also like to show that sophisticated academic music analysis techniques can be combined together and applied to real world problems in an efficient and reliable manner. 

Finally the project aims to be more than just a research study of feasibility. The result of successful completion will be an application of sufficient reliability and quality that it can be released to, and used by, untrained computer users. To our knowledge, it is the only computer game allowing people to generate Guitar Hero-like levels that also generates the surroundings tailored to every music track.

