% Chapter Template

\chapter{Abstract} % Main chapter title

\label{Abstract} % Change X to a consecutive number; for referencing this chapter elsewhere, use \ref{ChapterX}

\lhead{Abstract} % Change X to a consecutive number; this is for the header on each page - perhaps a shortened title

Music games present a highly pervasive new platform to create, perform and appreciate music. In case of music rhythm games, very often the player is limited to the songs preprocessed by the game developers. In this project we attempt to create a music rhythm game which, given a music track, extracts its features to generate a level without human intervention.

We explore the possibility of using main melody extraction to lead the generation of the main task for the user. In addition to this, we investigate a relationship between musical features and a mood perceived in a song to train a neural network capable of predicting arousal and valence values of a melody. Finally, we design and implement a system to detect boundaries between song segments and label them in an easy to understand way.

This report details the design of such a program and evaluates its effectiveness. The development of the program has lead to the discovery of new and powerful algorithms in music analysis, as well as successfully demonstrating the power of computers in developing creative works.


