% Chapter Template

\chapter{Design and Implementation} % Main chapter title

\label{Chapter5} % Change X to a consecutive number; for referencing this chapter elsewhere, use \ref{ChapterX}

\lhead{Chapter 5. \emph{Design \& Implementation}} % Change X to a consecutive number; this is for the header on each page - perhaps a shortened title


In this chapter we will go over the implementation process of the project. We will describe various choices we made, justifying them in the context of our objectives. 

First we will describe our solution to the mood detection problem. We first try to determine which musical features are the most correlated to the AV values of the music's emotion. Then, by training a neural network with data containing chosen features we will create a way of determining the arousal and valence value of any musical track, which will be later used in the implementation of our game. In addition to this, by investigating the impact of different parameters we will make sure out network has as good performance and accuracy as possible.

Next, we will move on to main melody detection by looking at two algorithms - one using source separation based approach and the other using the salience based approach. We will evaluate performance of both of them on data from recent pop culture to determine their performance and fitness in this project.

The next section will describe our attempt to automated music segmentation.

Last, but not least, we will talk about the game itself, its architecture, flow of use and design choices made.

\vspace{20pt}

\section{Mood Detection}

A common reason for engaging in music listening is that music is an effective means of conveying and evoking emotions. Although they may be subjective, based in part on the listener’s cultural and musical background or preferences, there are commonalities in perceived emotion across different listeners based on the characteristics of the music. Several studies have attempted to predict emotion conveyed during music listening. In our approach, we decided to represent the emotion connected to the music using a two-dimensional space with valence on the x-axis and arousal on the y-axis, first proposed by R. E. Thayer \cite{Thayer}.

As we described in Section \ref{sec:emotionClass}, there is a relation between valence and arousal values for a musical track and the moods perceived by people. In essence, the high arousal is connected to how energetic the music is, whereas valence refers to how positive (or negative) the emotions in the track are. 

\vspace{10pt}

\subsection{Choice of Features}
Using Essentia library \cite{essentia}, we implemented an extractor to retrieve certain features from a song, which we would expect to have certain impact on the perceived mood of a musical piece:

\begin{description}

\item[average loudness] - dynamic range descriptor. It rescales average loudness into the [0,1] interval on a per window basis. The value of 0 corresponds to signals with large dynamic range, 1 corresponds to signal with little dynamic range. This could indicate the level of the arousal, with higher loudness implying higher arousal value. We believe this relation could be quite intuitive - sad or peaceful songs tend to be quiet whereas excited or angry emotions are usually linked to louder tracks.

\item[means and derivatives of variance of rates of silent frames] in a signal for thresholds of 20, 30 and 60db. We believe that the values could influence the arousal levels, as the more and the bigger the silent gaps, the sadder / more peaceful the track seems to be, implying the low arousal value. When examining multiple musical tracks we have noticed that the happier or angrier songs can also have such silent gaps, but they tend to be much shorter.

\item[dynamic complexity] - computed on 2 second windows with 1 second overlap. The dynamic complexity is the average absolute deviation from the global loudness level estimate on the dB scale. It is related to the dynamic range and to the amount of fluctuation in loudness present in a recording. We believe this feature would have an impact on both examined values. However, similarly to the loudness level, arousal should be influenced more - as more dynamic songs (excited or angry) are more likely to suffer from loudness changes, whereas more phlegmatic ones (sad or peaceful) tend to keep the same dynamic complexity level.

\item[BPM] - beats per minute value according to detected beats. This feature should be correlated with the arousal level - intuitively, the faster the song, the more energetic it seems. 

\item[spectral centroid] - centroid statistics describing the spectral shape. It indicates where the ``center of mass'' of the spectrum is. Perceptually, it has a robust connection with the impression of ``brightness'' of a sound - an indication of the amount of high-frequency content in a sound. Timbre researchers consider brightness to be one of the perceptually strongest distinctions between sounds \cite{timber}, and formalise it acoustically as an indication of the amount of high-frequency content in a sound. That is why we believe the spectral centroid might be related to both valence and arousal.

\item[spectral RMS] (root mean square) - in physics it is a value characteristic of a continuously varying quantity, such as a cyclically alternating electric current or a sound. It is obtained by taking the mean of the squares of the instantaneous values during its duration or a cycle. This is linked to the loudness of the sound. This is why we believe that it might have an impact on arousal, but we do not exclude its impact on valence.

\item[spectral energy] - the energy E{s} of a continuous-time signal x(t) defined as: 
\begin{equation}
E{s}  =  \langle x(t), x(t)\rangle =  \int_{-\infty}^{\infty}{|x(t)|^2}dt
\end{equation}

Signal energy is always equal to the summation across all frequency components of the signal's spectral energy density. 
There have been some research focusing on relation between spectral energy and singing voice. In particular, in their paper \cite{spectralenergy}, S. Ferguson, D. T. Kenny and D. Cabrera were investigating the relation between the value and the experience of male singers. This makes for an interesting case worth considering in our research.

\item[mean and derivative of variance of beat loudness] -  spectral energy computed on beats segments of audio across the whole spectrum, and ratios of energy in 6 frequency bands. We suspect that the low value of the beat loudness could imply a low arousal.

\item[key and its scale] estimated key and its scale (major or minor) using Temperley’s profile. 
In music theory, the term key is used in many different and sometimes contradictory ways. A common use is to speak of music as being 'in' a specific key, such as ``in the key of C major or in the key of F\#''. Sometimes the terms 'major' or 'minor' are appended, as 'in the key of A minor' or 'in the key of B major'.
Broadly speaking the phrase 'in key of C' means that C is music's harmonic centre or tonic (the first degree of the scale, or the root of the scale). 
The terms 'major' and 'minor' further imply the use of a major scale or a minor scale. Thus the phrase 'in the key of E major' implies a piece of tonal music harmonically centred on the note E and making use of a major scale whose first note, or tonic, is E. 
We believe that those features can have an impact on both arousal and valence - songs performed in minor scale are traditionally connected to being sad, whereas the major scale is usually linked to positive emotions.

\item[scale and key of the chords] taken as the most frequent chord, and scale of the progression, whether major or minor. Scale commonly known to have a big influence on our perception on music \cite{keys}. It seems to be mostly the result of cultural conditioning as when people listen to tunes, they rely heavily on their memory. Such constant stimulus to our musical memory helps to generate expectations of what might come next in a tune or preserve the sound - emotion relation.

\item[means of zero-crossing rate] - the rate of sign-changes along a signal, i.e., the rate at which the signal changes from positive to negative or back. This feature has been used heavily in music information retrieval, being a key feature to classify percussive sounds. We believe it could be related to the arousal value. Music has a fairly normal distribution of frames with lower and higher zero-crossing rates. Speech however displays a much more skewed distribution. This could have an impact on songs where the vocals are quite rapid and energetic, for example rap music, and therefore might have a significant impact on mood recognition in our system.
ZCR is defined formally as: 
\begin{equation}
ZCR = \frac{1}{T-1} \sum_{t=1}^{T-1} {{\mathbb I}\left\{{s_t s_{t-1} < 0}\right\}}
\end{equation}

\item[pitch salience of a spectrum] - given by the ratio of the highest auto correlation value of the spectrum to the non-shifted auto correlation value.  Unpitched sounds (non-musical sound effects) and pure tones have an average pitch salience value close to 0 whereas sounds containing several harmonics in the spectrum tend to have a higher value. We think the value could have an effect on both the valence and arousal as pitch salience is often described as the probability of noticing a tone or clarity or strength of tone sensation.

\item[mean and derivative of variance of sensory dissonance] (to distinguish from musical or theoretical dissonance) of an audio signal given its spectral peaks. Sensory dissonance measures perceptual roughness of the sound and is based on the roughness of its spectral peaks. Given the spectral peaks, the algorithm estimates total dissonance by summing up the normalised dissonance values for each pair of peaks. These values are computed using dissonance curves, which define dissonance between two spectral peaks according to their frequency and amplitude relations. Dissonance could be related to low valence.

\end{description}

\begin{figure}
        \centering
        \begin{subfigure}[b]{0.48\textwidth}
                \includegraphics[width=\textwidth]{Figures/spectralcentroid-valence}
                \caption{A graph representing a correlation between spectral centroid and valence values.}
        \end{subfigure}%
        ~ %add desired spacing between images, e. g. ~, \quad, \qquad, \hfill etc.
          %(or a blank line to force the subfigure onto a new line)
        \begin{subfigure}[b]{0.48\textwidth}
                \includegraphics[width=\textwidth]{Figures/zerocrossing-arousal}
                \caption{A graph representing a correlation between zero-crossing rate and arousal values.}
        \end{subfigure}
          \caption{Chosen results of bivariate correlation with multiple regression.}
        ~ %add desired spacing between images, e. g. ~, \quad, \qquad, \hfill etc.
                        \label{fig:bivariateNN}

\end{figure}

\vspace{10pt}

\subsection{Correlation Between Features and Mood Perception}

In our exploration we decided to base our research on data collected in ``1000 Songs for Emotional Analysis of Music'' music library \cite{1000songs}, to avoid personal bias in assessing the mood of the song. The songs in the dataset were annotated by more than 300 crowdworkers on Amazon Mechanical Turk. Each song was annotated for arousal and for valence separately.

As a first step towards understanding the pattern by which audio features might account for emotion ratings, we conducted correlational analyses between features and mean valence/arousal ratings from the data set. We performed a bivariate correlation analysis with the valence/arousal ratings as the dependent variable, and each of the 22 features as the explanatory variable. Example of the results we achieved can be seen in Figure \ref{fig:bivariateNN}, the rest are included in Appendix A (Chapter \ref{AppendixA}, Section \ref{sec:bivariatediagram}) for reference. 

We found significant correlation between \textbf{valence} and derivative of variance and mean \textit{silence60}, derivative of variance of \textit{silence30, dynamic complexity, spectral centroid, spectral RMS, spectral energy, zero-crossing rate, pitch salience, and both mean and derivative of variance (dvar) of dissonance}. 

For \textbf{arousal}, we noticed correlation with \textit{spectral centroid, pitch salience, zero-crossing rate}, both \textit{mean} and \textit{dva}r of  \textit{silence60, spectral energy, mean dissonance} and \textit{dynamic complexity}. 

Values of all the features were then normalised between 0 and 1 to prepare them for the neural network training. 

\begin{wrapfigure}{l}{0.5\textwidth}
  \vspace{-30pt}
  \begin{center}
    \includegraphics[width=0.5\textwidth]{Figures/myANN}
  \end{center}
  \caption{A diagram depicting the structure of our artificial neural network for mood detection.}
\label{fig:finalnetwork}
\end{wrapfigure}

\vspace{10pt}

\subsection{Neural Network for Mood Prediction}

Our goal was to train the network to predict mean participant valence and arousal values for musical excerpts. 
Our first network implementation was a supervised, feedforward network with backpropagation. 
The input consisted of normalised values of 8 features:
\textit{spectral centroid, pitch salience, zero-crossing rate, silence60 mean  and dvar, mean dissonance, dynamic complexity} and \textit{spectral energy}. 
The network had two outputs - arousal and valence.

As all the training data was normalised, the input and output values were within a range of 0 to 1. The training set consisted of 50 input and output arrays. Each input array had 8 values, one per audio feature, and its corresponding output array had the two desired arousal and valence values.

The network’s task was to provide the valence and arousal values based on the 13 audio features. The output values fell within a range of 0 to 1. Since desired outputs were average valence/arousal ratings provided by participants on a scale from 1 to 9, the network outputs were rescaled back. The training set consisted of eight input and output arrays. Each input array had 13 values, one for each audio feature, and its corresponding output array had the two desired arousal and valence values. The connection weights from input to the hidden nodes and from hidden nodes to the output ones were initialised to random numbers. 

The network was built, trained, and tested using the pyBrain python library for neural network implementation. 

We trained our network for 1000 epochs with many different sizes of the hidden layer and default values for all the other parameters. The performance based on that can be seen in Table \ref{table:rsmetable}.


Hidden neurons are the neurons that are neither in the input layer nor the output layer. Using additional layers of hidden neurons enables greater processing power and system flexibility at the cost of additional complexity in the training algorithm. Having too many hidden neurons can be thought of as a system of equations with more equations than there are free variables: the system is over specified and incapable of generalisation. Having too few hidden neurons, conversely, can prevent the system from properly fitting the input data, and reduces the robustness of the system.

\begin{figure}[h]
	\centering
   \includegraphics[width=0.7\textwidth]{Figures/nodesperf}
\caption{Data presented in table \ref{table:rsmetable}, plotted on a diagram.}
\end{figure}


\begin{table}
\begin{center}
\begin{tabular}{| c | l | l | l | l |} \hline 
  No. of Nodes & RMSE 1 & RMSE 2 & result 1 & result 2  \\ \hline \hline
  1 &  0.0727638005274 & 0.0740582536152  & 0.0998088934575 & 0.0978822145006 \\ \hline
  2 &  0.071796654024 & 0.0709793303052 & 0.113046836083 & 0.112405435125 \\ \hline
  3 & 0.0722212571658 & 0.0733605257262 &  0.155412522783 & 0.0948392717258 \\ \hline
  4 &   0.0702013899702 & 0.0699921976435 &  0.102602437509 & 0.115373051966 \\ \hline
  5 &  0.0659433293266 &  0.0693361162261 &  0.116558760273  & 0.10423200269\\ \hline
  6 & 	 0.0722427034758 & 0.0751383013205 &  0.142248432275  &  0.118843096333  \\ \hline
  7 &  0.0698701385354  & 0.0678483277007  & 0.088954243616 & 0.103537259056 \\ \hline
  8 &   0.0692459138916 &  0.066424019477 & 0.131412928439 & 0.136098090028 \\ \hline
  9 &   0.0676910853628 &  0.0707274913708 & 0.128139548772 & 0.104231713578\\ \hline
 10 & 0.0684398705278 & 0.0673887116962 & 0.140505458102 & 0.175156506583\\ \hline
 15 & 0.0671656450239 & 0.0673141803371 & 0.116563143115  &  0.139265837027\\ \hline
 20 & 0.0737978227013 &  0.0720620813131 & 0.160424096589 & 0.136210925296\\ \hline
 50 & 0.0669456166054 &   0.0694139442297 & 0.164132829293 & 0.171603556123\\ \hline
\end{tabular}
\caption{Table showing the root mean square error for training the network for given number of nodes in the hidden layer.}
\label{table:rsmetable}
\end{center}
\end{table}


As we can see, the optimal solution is the one with 7 nodes in the hidden layer. Although the initial RMSE returned after training is not overall minimum, all the values - so both the training ones and the ones after the evaluation, are local minimas and one of the minimal values overall. This decision can be justified by the fact that although for some cases we managed to achieve smaller RSME from the training, the network was in fact overfitting, and doing really well for the already known input, but worse for a new one.
To avoid overfitting the network, we kept the number of hidden units equal to the number of input units. 

\begin{table}
\begin{center}
\begin{tabular} {| c | l | l |} \hline
 Learning Rate & RMSE & result RMSE \\  \hline \hline
 0.3 & 0.0707970888752 & 0.14578838717 \\ \hline
 0.25 & 0.0699336891245 & 0.163193322998 \\ \hline
 0.2 &  0.0667986974361 & 0.15521882498 \\ \hline
 0.15 & 0.0724218948598 & 0.104971086068 \\ \hline
 0.1 & 0.0684257582616 & 0.100719004205 \\ \hline
 0.05 & 0.0695957657331 & 0.0979349713899 \\ \hline
 0.01 & 0.0689460348924 & 0.090954243616 \\ \hline
 0.005 & 0.0724023992534 & 0.130733683966 \\ \hline
 0.001 &  0.079664786995 & 0.112619882406 \\ \hline
\end{tabular}
\caption{Table showing the root mean square error for training the network for given learning rate parameter value.}
\label{table:learningrate}
\end{center}
\end{table}

\begin{figure}[h]
	\centering
   \includegraphics[width=0.7\textwidth]{Figures/learningrate}
\caption{Data presented in table \ref{table:learningrate}, plotted on a diagram.}
\end{figure}


Having found the optimal number of nodes in the hidden layer, we moved on to find the learning rate parameter. Training parameter that controls the size of weight and bias changes in learning of the training algorithm. In a standard backpropagation, too low a learning rate makes the network learn very slowly,whereas a learning rate that is too high makes the weights and objective function diverge, so there is no learning at all. 

We started our search by setting it to 0.3 and reducing it over time. The results we found can be found in Table \ref{table:learningrate}. As we can see, the optimal solution seems to be learning rate at value 0.001.


In the end, we came up with the network which can be seen on Figure \ref{fig:finalnetwork}.

\vspace{20pt}

\section{Main Melody Extraction}

\vspace{10pt}

\section{Structure Retrieval}

Understanding the structure of music (e.g. intro, verse, chorus, bridge, and outro) is important as it allows us to divide a song into semantically meaningful segments, within which musical characteristics are relatively consistent.

\vspace{10pt}

\subsection{Feature Choice}

To implement a system capable of unsupervised structure recognition, we need to provide it with some data. We investigated two possible values - \textit{Mel-frequency cepstral coefficients} and \textit{harmonic pitch class profile}.

Cepstrum is the result of taking the Inverse Fourier transform (IFT) of the logarithm of the estimated spectrum of a signal. It can be viewed as information about rate of change in the different spectrum bands.

The mel-frequency cepstrum (MFC) is a representation of the short-term power spectrum of a sound, based on a linear cosine transform of a log power spectrum on a nonlinear mel scale of frequency.

\textit{Mel-frequency cepstral coefficients} (MFCCs) are coefficients that collectively make up an MFC. They are increasingly finding uses in music information retrieval applications such as genre classification, audio similarity measures, etc.

MFCCs are derived from a type of cepstral representation of the audio clip. The mel-frequency cepstrum differs from cepstrum by having its frequency bands equally spaced on the mel scale, which approximates the human auditory system's response more closely than the linearly-spaced frequency bands used in the normal cepstrum. This frequency warping can allow for better representation of sound, for example, in audio compression.

MFCCs are commonly derived as follows:
\begin{itemize}
\item Take the Fourier transform of a signal.
\item Map the powers of the spectrum obtained above onto the mel scale, using triangular overlapping windows.
\item Take the logs of the powers at each of the mel frequencies.
\item Take the discrete cosine transform of the list of mel log powers, as if it were a signal.
\item The MFCCs are the amplitudes of the resulting spectrum.
\end{itemize}

An alternative to using MFCCs as the features to base the algorithm on is \textit{HPCP}.

Harmonic pitch class profiles (HPCP) is a vector of features extracted from an audio signal, based on the Pitch Class Profile descriptor. HPCP is an enhanced pitch distribution feature which is a sequence of chroma - feature vectors describing tonality measuring the relative intensity of each of the 12 pitch classes of the equal-tempered scale within an analysis frame. 

HPCP features can be found and used to estimate the key of a piece, to measure similarity between two musical pieces and to classify music in terms of composer, genre or mood. The process is related to time-frequency analysis. In general, chroma features are robust to noise, for example an ambient noise or percussive sounds, independent of timbre and instrumentation and independent of loudness and dynamics.

The General HPCP feature extraction procedure is summarised as follows:
\begin{itemize}
\item Input musical signal.
\item Do spectral analysis to obtain the frequency components of the music signal.
\item Use Fourier transform to convert the signal into a spectrogram. (The Fourier transform is a type of time-frequency analysis.)
\item Do frequency filtering. A frequency range of between 100 and 5000 Hz is used.
\item Do peak detection. Only the local maximum values of the spectrum are considered.
\item Do reference frequency computation procedure. Estimate the deviation with respect to 440 Hz.
\item Do Pitch class mapping with respect to the estimated reference frequency. This is a procedure for determining the pitch class value from frequency values. A weighting scheme with cosine function is used. It considers the presence of harmonic frequencies (harmonic summation procedure), taking account a total of 8 harmonics for each frequency. In order to map the value on a one-third of a semitone, the size of the pitch class distribution vectors has to be equal to 36.
\item Normalise the feature frame by frame dividing through the maximum value to eliminate dependency on global loudness.
\end{itemize}

The discussion of results given each of the alternatives are discussed in Section \ref{sec:structurefeatures}.

\vspace{10pt}

\subsection{Feature Preparation}

In this section, we will describe the process of preparation of the features for improving the performance of the algorithm. For simplicity and clarity, when talking about the features, we will first focus on analysis based on HPCPs, followed by one on MFCCs.
We decided to investigate both possibilities as they present the track from completely different perspective. For example, the HPCP chroma might fail to distinguish vocal and instrumental parts if the underlying harmonic patterns are exactly the same. On the other hand, when working with MFCCs we expect the opposite behaviour - good performance on parts that are different in terms of timbre.

\subsubsection*{Harmonic Pitch Class Profiles}

\begin{figure}
        \centering
        \begin{subfigure}[b]{0.47\textwidth}
                \includegraphics[width=\textwidth]{Figures/hpcp_unsynched_chroma}
                \caption{Example of a chromagram without any further enhancement. }
                \label{fig:unchroma}
        \end{subfigure}%
        ~ %add desired spacing between images, e. g. ~, \quad, \qquad, \hfill etc.
          %(or a blank line to force the subfigure onto a new line)
        \begin{subfigure}[b]{0.47\textwidth}
                \includegraphics[width=\textwidth]{Figures/hpcp_synched_chroma}
                \caption{Example of a chromagram after beat-synchronisation.}
                \label{fig:synchroma}
        \end{subfigure}
          \caption{Harmonic pitch class profiles chroma features calculated for a song by The Beatles- ``Help!''.}
        \label{fig:chromacomparison}
\end{figure}

A series of transformations are applied to the data in order to distinguish the different parts of a song more efficiently with preserving the accuracy.. 

First, we need to we synchronise our data with the beats detected in the musical track. This process allows to reduce local variation by summarising (usually taking the mean or a median) frame-wise features that occur between two beats, yielding fewer but longer beat-synchronous frames. The rationale for doing so is that many features, such as chord labels that occur between two consecutive beats tend to be the same. Thanks to focusing on the values of the features on a per-beat basis, we manage to largely normalise variations in tempo. However, the main advantage of applying the beat-synchronisation is that we manage to reduce the amount of data to analyse, and hence, the size of the matrix, we are operating on.

This leads to beat-synchronous chromagrams. A diagram of a chromagram after beat synchronisation can be seen in Figure \ref{fig:synchroma}. As we can see, the size has decreased dramatically, which makes the segmentation process computationally cheaper.

\begin{figure}
        \centering
        \begin{subfigure}[b]{0.47\textwidth}
                \includegraphics[width=\textwidth]{Figures/hpcp_synched_log_chroma}
                \caption{Chroma feature after applying log normalisation.}
                \label{fig:logchroma}
        \end{subfigure}%
        \begin{subfigure}[b]{0.47\textwidth}
                \includegraphics[width=\textwidth]{Figures/hpcp_synched_median_chroma}
                \caption{Chroma feature after applying sliding median filter of size h=9.}
                \label{fig:slidingchroma}
        \end{subfigure}
          \caption{Beat-synchronised chroma created for song ``Help!'' by The Beatles with applied enhancements.}
        \label{fig:chromaenhance}
\end{figure}

Following the beat-synchronisation, we apply log normalisation to the chroma feature. This allows us to reduce the effect the outliers from the trend will have and further improve the contrast between the related and unrelated beat frames. The enhancement achieved by applying log normalisation can be seen in Figure \ref{fig:logchroma}.

As the next step, we applied a sliding median filter of size h is run against each of the beat-synchronous and log-normalised chromagram channels, which can be seen in Figure \ref{fig:slidingchroma}. Thanks to the median filter, we can come up with sharper edges than with a regular mean filter. This becomes really useful in obtaining section boundary precision.

\begin{figure}
        \centering
        \begin{subfigure}[b]{0.47\textwidth}
                \includegraphics[width=\textwidth]{Figures/hpcp_unsynched_ssm}
                \caption{Similarity matrix generated from harmonic pitch class profiles chroma without further enhancement.}
                \label{fig:unSSM}
        \end{subfigure}%
        \begin{subfigure}[b]{0.47\textwidth}
                \includegraphics[width=\textwidth]{Figures/log_ssm_synched}
                \caption{Similarity matrix generated from beat-synchronised harmonic pitch class profiles chroma.}
                \label{fig:synSSM}
        \end{subfigure}
          \caption{Comparison of SSM generated from unprocessed and enhanced chromas, using correlation distance.}
        \label{fig:ssmcomparison}
\end{figure}

By filtering features across time, we retain the most prominent chromas within the h-size window and remove smaller artefacts, which are irrelevant in our con- text. The Figure \ref{fig:slidingmedian} presents the chromagram after applying the sliding median filter.

We then proceed to compute the Self Similarity Matrix (SSM) of the pre-filtered beat-synchronous chromagram. The SSM is essentially a pairwise comparison of a given set of features using a specific distance measure between the features of the two beat indices i and j. The result of every such comparison is stored in a N x N symmetric matrix D, such that D(i, j) contains said distance. In particular, D(i, j) stores the same value ad D(j, i), and for every i D(i, i) is equal to 0.

We investigated the influence of the type of the distance calculated on the SSM produced for the enhanced chroma. In our research we looked into four types of distance: euclidean, manhattan, correlation and cosine. Our results are presented in Figure \ref{fig:ssmdistance}. 
As we can see in Figures \ref{fig:euclidean} and \ref{fig:manhattan}, the contrast achieved is much weaker. Not only there are fewer blue spots signifying small or even no similarity between points, but the amount of points that are significantly similar is also reduced.


\begin{figure}[b]
        \centering
        \begin{subfigure}[b]{0.31\textwidth}
                \includegraphics[width=\textwidth]{Figures/ssm_euclidean}
                \caption{Similarity matrix calculated from harmonic pitch class profiles chroma using Euclidean distance.}
                \label{fig:euclidean}
        \end{subfigure}%
        \begin{subfigure}[b]{0.31\textwidth}
                \includegraphics[width=\textwidth]{Figures/ssm_manhattan}
                \caption{Similarity matrix calculated from harmonic pitch class profiles chroma using Manhattan distance.}
                \label{fig:manhattan}
        \end{subfigure}
         \begin{subfigure}[b]{0.31\textwidth}
                \includegraphics[width=\textwidth]{Figures/ssm_cosine}
                \caption{Similarity matrix calculated from harmonic pitch class profiles chroma using cosine distance.}
                \label{fig:cosine}
        \end{subfigure}
          \caption{Comparison of SSM computed using different distance formulas. The SSM calculated using correlation distance can be seen in Figure \ref{fig:synSSM}.}
        \label{fig:ssmdistance}
\end{figure}


When we look at the SSM computed using cosine distance, we can notice that the amount of the similar points has increased, more similar to the the one generated using the correlation distance. However, the correlation distance on Figure \ref{fig:synSSM} contains more dark blue spots, implying that it exposes more beats that are, in fact, not similar. This is why, in our design of the structure retrieval of a song we decided to use SSM computed using correlation distance.

\vspace{10pt}

\subsubsection*{Mel-frequency Cepstral Coefficients}

\begin{figure}
        \centering
        \begin{subfigure}[b]{0.47\textwidth}
                \includegraphics[width=\textwidth]{Figures/mfcc_no_log_sync}
                \caption{Similarity matrix generated from Mel-frequency Cepstral Coefficients without log normalisation.}
                \label{fig:unMFCC}
        \end{subfigure}%
        \begin{subfigure}[b]{0.47\textwidth}
                \includegraphics[width=\textwidth]{Figures/mfcc_ssm_synched}
                \caption{Similarity matrix generated from Mel-frequency Cepstral Coefficients with application of log normalisation.}
                \label{fig:synMFCC}
        \end{subfigure}
          \caption{Comparison of SSM generated from unprocessed and enhanced MFCCs, using correlation distance.}
        \label{fig:MFCCcomparison}
\end{figure}

Similarly to the case of Harmonic Pitch Class Profiles, we start the preparation of the features by beat-synchronisation to decrease the size of the data for further analysis. 

Now we have to determine whether log normalisation will improve the clarity of the SSM. As we can see in Figure \ref{fig:MFCCcomparison}, the use of log normalisation could decrease the amount of segments found, as more similar points are exposed.

Finally, we compute the SSM. Again, we investigated the possibility of generating it using Euclidean, Manhattan and cosine distances. The diagrams presenting our findings can be seen in Appendix B (\ref{AppendixB}). Similarly to when we were working with HPCPs, the correlation distance gave us the most contrasted, sharper images. 

The result of this process can be seen in Figure \ref{fig:synMFCC}.

\vspace{10pt}

\subsection{C-NMF}

We can view the SSM as an array of column vectors where each vector corresponds to a window. Suppose we have a set of vector templates. Vectors in the steady regions of a song may be directly found in the set, while vectors in the boundary regions may be approximated by linear combination of vector templates. Making this observation, we believe the Non-negative Matrix Factorization (NMF) could be useful in our situation.

\begin{wrapfigure}{l}{0.5\textwidth}
  \begin{center}
    \includegraphics[width=0.45\textwidth]{Figures/NMF}
  \end{center}
  \caption{Illustration of approximate non-negative matrix factorization: the matrix X is represented by the two smaller matrices W and H.}
\label{fig:NMF}
\end{wrapfigure}

In NMF, the $N \times N$ self similarity matrix $X$ is approximately factorised into product of a $N \times k$ matrix $W$, can be interpreted as a cluster row matrix, and $k \times N$ matrix $H$, composed of the indicators of these clusters, where \textit{k} is the rank of the composition. This can be described as $X \approx WH$. The \textit{j}th column of $W$ can be viewed as the vector template for the \textit{i}th segment type. The \textit{j}th column of $H$ describes the intensities of the \textit{k}th segment types for the \textit{j}th window. In NMF, both $W$ and $H$ are enforced to be positive (i.e. $X$ must be positive too). We denote a row vector by $\boldsymbol{z}$ and a column one by $\boldsymbol{z}^{T}$.

However, in data mining, sometimes it can be beneficial to ensure X to contain meaningful “cluster centroids”, i.e., to restrict W to be convex combinations of data points.
C-NMF adds a constrain to $W$ = ($\boldsymbol{w}_{1}^{T}$, $\boldsymbol{w}_{2}^{T}$,... ,  $\boldsymbol{w}_{k}^{T}$),  such that its columns  $\boldsymbol{w}^{T}$  are, in fact,  convex combinations of the features of $X$:

\begin{equation}
\boldsymbol{w}_{j}^{T} = \boldsymbol{x}_{1}^{T}f_{1j} + \boldsymbol{x}_{2}^{T}f_{2j} + ... + \boldsymbol{x}_{N}^{T}f_{Nj}  \hspace{45pt}   j \in [1 : k]
\end{equation}

The linear combination is convex if all coefficients $f_{ij}$ are positive and the sum of each set of coefficients  $\boldsymbol{f}^{T}_{j}$ must be 1. Formally, this can be represented as:
$f_{ij} \geq 0,  f_{ij} = 1 $

This results in $W = XF$, where $F \in \mathbb{R}^{N \times k}$, which makes the rows $\boldsymbol{f}_{i}$ interpretable as weighted cluster centroids. The decomposition matrices $R_{j}$, are obtained as follows:  $R_{j} =  \boldsymbol{w}^{T}_{j}\boldsymbol{h}_{j}$, where $j \in [1 : k]$. Finally, C-NMF can be formally characterised as: $X \approx XFH$.

In C-NMF, the matrix W is a set of convex combinations of the rows of the input matrix X, which contrasts with NMF, where no such constraint exists. This means that, each row $x_{i}$ represents similarity of the time frame i with the rest of the time frames, storing information about the time frae i across the entire song.

By computing the C-NMF we separate basic structural parts. In the next sec- tion, we describe how the factorization via C-NMF relates to structure and show how we can use that result for music structure discovery.   ============ change ============

==== maybe try to get the plots of the decomposition matrices? =====

Apart from this, another important benefit of C-NMF over NMF is that matrices $F$ and $G$ become naturally sparse when adding the convex constrain. In case of NMF the $G$ does not always become sparse. Thanks to that, when using C-NMF we are more likely to find similar decomposition matrices for the same input than NMF, which is more sensitive to its initialisation \cite{Nieto}. 

\vspace{10pt}


\subsection{Boundaries}

\begin{figure}
        \centering
        \begin{subfigure}[b]{0.47\textwidth}
                \includegraphics[width=\textwidth]{Figures/F}
                \caption{The cluster matrix W.}
                \label{fig:Wmatrix}
        \end{subfigure}%
        \begin{subfigure}[b]{0.47\textwidth}
                \includegraphics[width=\textwidth]{Figures/G}
                \caption{The activation matrix H.}
                \label{fig:Hmatrix}
        \end{subfigure}
          \caption{The result of C-NMF computed for ``Help!'' by The Beatles with rank k = 3.}
        \label{fig:CNMFbeatles}
\end{figure}
 
In this section we will investigate different ways of obtaining section boundaries from the decomposition matrices obtained from applying C-NMF to the similarity matrix. An example of resulting cluster and activation matrix computed with C-NMF with rank k = 2 can be seen in figure \ref{fig:CNMFbeatles}.


\subsection*{K-means Clustering}

\begin{figure}[b]
        \centering
        \begin{subfigure}[t]{0.30\textwidth}
                \includegraphics[width=\textwidth]{Figures/underfitting}
                \caption{The k value is too small - many long distances to centroids.}
                \label{fig:euclidean}
        \end{subfigure}%
        \begin{subfigure}[t]{0.30\textwidth}
                \includegraphics[width=\textwidth]{Figures/justright}
                \caption{Good k value - distances to centroids are quite short.}
                \label{fig:manhattan}
        \end{subfigure}
         \begin{subfigure}[t]{0.30\textwidth}
                \includegraphics[width=\textwidth]{Figures/overfitting}
                \caption{The k value is too big - little improvement in average distance.}
                \label{fig:cosine}
        \end{subfigure}
          \caption{Diagrams depicting impact of the k value on the clustering result \cite{kcluster}.}
        \label{fig:ssmdistance}
\end{figure}

Clustering is an unsupervised classification of patterns, for example observations, data items, or feature vectors, into groups called clusters. The points within each cluster should be similar to each other and dissimilar to points belonging to another cluster. The problem has been addressed in many contexts and by researchers in many disciplines.

K-means clustering is considered one of the simplest unsupervised learning algorithms that can solve the well know clustering problem. It follows a simple and easy way of classifying a given data set through a certain number of clusters (assume k clusters).

The main idea  is to define k centres, one for each cluster. Much care should be put into their placement, as different location of centres causes different result. This is why the most intuitive solution is to put them as far apart as possible. The next step is to take one point after another from the data set and associate it with the nearest centre, 

When all the points have been assigned to some centre, k new centroids are calculated as baycentres of the clustering that resulted from the first phase. Once we have calculated the new centroids, a new binding has to be done  between the same data set points and the nearest new center. Very often this will result in points moving between different clusters. This can be considered second phase of the algorithms. 

Those two phases are repeated in a loop. As a result, we may notice that the k centers change their location step by step until no more changes are done, and the algorithm converges. 

We ran k-means clustering with k = 2 to each one of the C-NMF decomposition matrices, interpreting them as row-vector features. We efficiently obtained the section boundaries The choice of k = 2 allows us to detect boundaries (i.e. there’s a boundary or not), regardless of how the various sections cluster. However, after comparing the output of an this algorithm with a manually created segmentation, we noticed that k-means clustering's performance was not suited for our use. The granularity of the segmentation was too high - very often it separated parts of verse, or even fractions of seconds. Even after merging  values that were close together and getting rid of the smallest segments, the boundaries detected were too granular.

 
\vspace{10pt}

\subsection*{Another Approach}
Having applied the C-NMF, we computed, we have generated two decomposition matrices - $W$, called the cluster matrix, and $H$, an activation matrix, so that by multiplying them we can recreate the SSM, ie. $X \approx  WH$       
        
To obtain the boundaries, we filter both the cluster matrix $W$ and the activation matrix $H$. This means that for every row corresponding to a frame in which the beat occurs, we find the indexes of the its maximum values. Thanks to this simple clustering, we obtain a assignment of each of the frames to one of the 'types' of segments. By iterating through a matrix generated and such ways and recording the indexes at which a song changes from one portion to another, we manage to obtain section boundaries in an efficient way. In our implementation, we start the computation with rank $k = 3$ and increase it if not enough bounds were found. This is to avoid overfitting and creating tiny segments, for example, one per verse line.

Once we have boundaries, we combine them within a distance window of size h so that boundaries close to each other get merged in their average location.

This method, although much simpler, produced much neater output, with fewer, but more meaningful segments.


\subsection{Labelling}

In our structure retrieval, we investigated different ways of labelling of the segments.

First approach built directly on our way of segmenting the song.
Similarly to when attempting to find boundaries between segments, we decomposed our self similarity matrix X with rank increased. This allowed further differentiation between structures, for instance, if certain part of a song was thought of as a chorus, but it is different enough to get separated in a C-NMF of a higher rank.

Once we have decomposed the matrix $X$, we filtered the clustering matrix $W$. This way, we computed initial clustering for each beat frame. To synchronise labels calculated in such way, we assigned a numerical label which occurred within each of the interval between two segment to them.


\begin{figure}[h]
	\centering
   \includegraphics[width=0.7\textwidth]{Figures/NumericalLabels}
\caption{A depiction of the pure vocals detection as a labelling method (bottom) compared with manual segmentation and labelling (top).}
\label{fig:vocalsimple}
\end{figure}

However, having gathered data about the main melody of the song earlier on, we could use it to more accurately predict the labels for each of the song segments, producing labels that easy to understand to a person. In attempt to do so, we synchronise the array of pitches with the beats to get beat synchronised features which we could with ease refer to once we have the boundary frames. 


\begin{figure}[h]
	\centering
   \includegraphics[width=0.7\textwidth]{Figures/BeatlesSegmentation}
\caption{A depiction of the pure vocals detection as a labelling method (bottom) compared with manual segmentation and labelling (top).}
\label{fig:vocalsimple}
\end{figure}

First intuition we had was to investigate the amount of silent frames occurring in each of the segments. We designed heuristics to decide whether the segment we look at is vocal (majority of the frames contain the main melody), semi-vocal (the same amount of frames that are contain the main melody and not), instrumental (minority of the frames contain the main melody) or silent. 
However, this way alone, we lose the distinction between the vocal parts, for instance, between the verse and the chorus.

A visualisation of our segmentation and labelling using only vocal/instrumental heuristics which can be seen on Figure \ref{fig:vocalsimple}. The diagram makes it evident that our boundary finding works quite well. As we can see, each beginning of the chorus is detected with high accuracy. Moreover, the beginning and the end of the intro and outro parts are predicted really well. In addition to this, the segmentation process manages to detect the silence segments in the beginning and end of track. However, our clustering algorithm detects the end of the chorus too early, absorbing a part of it into the verse that follows.





\vspace{20pt}




\section{The Game}

In this section, we will go over the architecture and the design choices made when planning and implementing our game.

The game is written mostly in Swift, a multi-paradigm, compiled programming language created by Apple Inc. for iOS and OS X development. 
It was first introduced at Apple's 2014 Worldwide Developers Conference (WWDC). Swift is designed to work with Apple's Cocoa and Cocoa Touch frameworks, building on the best of C and Objective-C, without the constraints of C compatibility. It adopts safe programming patterns and adds modern features to make programming easier, more flexible, and more fun \cite{swiftintro}. 

We chose this language as we wanted to create a game for the OS X platform. In addition to this, the author also had a personal interest in learning the language.

\vspace{10pt}

\subsection{Data Storage}

The game relies on preserving user's scores and the levels generated by them. We need a way of storing them and all the information retrieved when analysing the songs to avoid regenerating the levels for the same song, for example if the user has a music piece they particularly like.

Core Data is the standard way to persist and manage data in both iPhone and Mac applications. It is an object graph and persistence framework provided by Apple in the Mac OS X and iOS operating systems. 

Core Data describes data with a high level data model expressed in terms of entities and their relationships plus fetch requests that retrieve entities meeting specific criteria. Code can retrieve and manipulate this data on a purely object level without having to worry about the details of storage and retrieval. 

Core Data allows data organised by the relational entity–attribute model to be serialised into XML, binary, or SQLite stores.

Core Data is also a persistent technology, in that it can persist the state of the model objects to disk. But the important takeaway is that Core Data is much more than just a framework to load and save data - it is also about working with the data while it is in memory.
We decided to use Core Data rather than a separate database as our game only needs to store data used by the current user, that will be utilised almost immediately after loading into memory. 
  
The model might cause some intensive memory usage if we decide to create a big amount of users, however, as it is an offline game that can be played on a personal machine, in contrast to web application, the number of users should remain relatively small.

\vspace{10pt}

\subsection{Menu}

Although not usually adopted in OS X games, we decided to follow the Model-View-Controller design pattern in implementing our application. We believe it was a right choice as the complexity of the main menu would have to be then supported throughout the played level. This would not only be a performance strain, but would also cause the code to be messy.

When first facing the menu, the user has an option of creating an account, logging in as a user or playing a quick game, not requiring any user data. 
The quick game is essentially an ability of playing one of the predefined levels, without a choice of creating a new one.

Once the user has created an account or chosen an existing one, they can either follow the level creation or level loading option. If they choose to create a new level, they have to select a file from their hard drive they would like to use as the base for their level. Otherwise, they go to the window, where they can select a level and either play it or remove it from their catalogue.

\vspace{10pt}

\subsection{Level Description}

Once we move on to playing a game, the \verb|GameViewController| unpacks the \verb|GameScene| - an object representing a scene of content in Sprite Kit.

Sprite Kit provides a graphics rendering and animation infrastructure that can be used to animate arbitrary textured images, or sprites. It uses a traditional rendering loop where the contents of each frame are processed before the frame is rendered. Its advantage is that it was developed for Apple hardware, hence it is optimised to render frames of animation efficiently using the graphics hardware. Thanks to this, the positions of sprites can be changed arbitrarily in each frame of animation. Sprite Kit also provides other functionality that is useful for games, including basic sound playback support and physics simulation. \cite{spritekit}. 

In the game scene, there is a set of buttons at the bottom of the screen. Players use the strum bar along with the fret buttons to play notes that scroll down the screen. The Easy difficulty only uses the first three fret buttons, that is, the green, red, and yellow. The Medium difficulty uses the blue button in addition to those three, and Hard and Expert use all five buttons.

The score is calculated based on how many scrolling notes we manage to hit. Every time we hit, the performance bar on the right side of the screen goes up, otherwise it goes down. If it hits the minimum, it the player loses. However, if the player manages to keep the performance level at the maximum for an appropriate amount of time, the number of the points scored for the new notes gets doubled until he misses a note or wins the level.

The player can at any time pause, stop or replay the game. They can also control the volume of the music and other sounds in the game. 

Upon completion of the level the player presented with their score, shown as stars and a concrete number. The player can later revisit the levels if they want to improve their score. 

\vspace{10pt}

\subsection{Melody Detection as a Game Changer}

The main part of the gameplay relies on the user pressing buttons that line up on the screen. As this is a music game, there are various ways of making the process more intuitive and hence attractive to potential players. 

One of the possibilities is to use the main melody of the song to determine which buttons to issue for the player by tying in the melody extraction.


...

Having processed the list of pitches in such way, we have prepared the ground for the buttons generation.

The main idea of the game is to mimic the playing an instrument on the computer keyboard. For this purpose we looked to 2 most popular instruments - guitar and piano for inspiration. 

When playing the piano, we are presented with a set of keys. 

\vspace{10pt}

\subsection{Introduction of The Song Segmentation}

\vspace{10pt}

\subsection{Impact of the Mood on the Level}

\vspace{20pt}


\section{Main Section 2}
